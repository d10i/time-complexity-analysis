\begin{abstract}
  Profilers have been around in the field of software development since the 70’s
  and have been – and still are – very useful in diagnosing problems and as tools
  for programs optimisation. They can analyse how much time the program is
  spending in each method. Based on this information the developer is able to
  detect what part of the program is taking more time than expected to execute and
  can act to improve it.
                        
                        
  \noindent What profilers can’t do is to estimate how much time it would take when the input
  size is increased. This dissertation supports the development of a tool that will
  estimate the time complexity of an algorithm under test with a high precision. This
  allows accurately estimating the time needed for the algorithm to solve a problem
  for any input size, giving developers a precise idea of how the software will
  behave with input sizes that haven’t been profiled.
\end{abstract}
